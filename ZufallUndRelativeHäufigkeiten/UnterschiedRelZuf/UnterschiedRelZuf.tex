\documentclass[]{article}

\usepackage{amsmath}
\usepackage{amssymb}
\usepackage[margin = 1in]{geometry}

\begin{document}

\newcommand{\chanceOrRel}{\\\ \\$\square\text{ Relative Häufigkeit }\qquad\square\text{ Zufall}$\\ \begin{tabular}{c}
\ \\
\rule{\textwidth}{1pt}\\
\ \\
\rule{\textwidth}{1pt}\\
\end{tabular} }

\section*{Zufall oder Relative Häufigkeit?}
\textbf{Kreutzt an}: Handelt es sich bei der Aufgabe um eine relative Häufigkeit oder um eine Zufallsgröße? \textbf{Begründet} kurz.
\begin{itemize}
	\item Eine Tablette enthält 20mg Wirkstoff. Wie viel Wirkstoff wird für 60.000 Tabletten benötigt?
	\chanceOrRel
	\item Ein Roboterarm verpackt Bauklötze von einem Rüttelband aufrecht in eine Schachtel. Wie häufig muss der Arm einen Klotz drehen? (Wie viel Stress liegt auf diesem Gelenk?)
	\chanceOrRel
	\item In einem handelsüblichen Pkw hält der Zahnriemen des Motors im Schnitt 4 Jahre. Wie viele Zahnriemen benötigt man bei einer Lebensdauer des Wagens von 25 Jahren?
	\chanceOrRel
	\item Eine Sicherung muss laut Hersteller alle 2 Jahre gewechselt werden. Wie viele Sicherungen benötigt man während eines Projektes über 14 Jahre?
	\chanceOrRel
	\item Beim aussähen auf einem Feld schlagen ca. die Hälfte der Keimlinge Wurzeln. Wie viele Keimlinge werden benötigt, um ein Feld zu bestücken?
	\chanceOrRel
	\item Bei der Reaktion von Wasserstoff und Sauerstoff wird jeweils ein Sauerstoffatom und zwei Wasserstoffatome zu einem Dihydrogenoxyd-Molekül reagiert. Wie viele Sauerstoff- bzw. Wasserstoffatome werden für die Herstellung von $6\cdot 10^{23}$ Dihydrogenoxyd-Molekülen benötigt?
	\chanceOrRel
\end{itemize}

\end{document}