\documentclass{article}

\usepackage{multirow}
\usepackage{graphicx}
\usepackage{tabularx}
\usepackage{longtable}
\usepackage{ltablex}

\begin{document}
\section*{Unterrichtseinheit: Variablen in Scratch}


\begin{tabularx}{\textwidth}{|X|X|X|X|X|}
\hline
\multirow{2}{*}{Phase (Zeit)}    & Aktivitäten                                                                                                                 &                                                                                      & \multirow{2}{*}{\begin{tabular}[c]{@{}l@{}}Sozial- und \\ Interaktionsform\end{tabular}} & \multirow{2}{*}{\begin{tabular}[c]{@{}l@{}}Material/\\ Medien\end{tabular}} \\ \cline{2-3}
                                 & Lehrkraft                                                                                                                   & Schüler*innen                                                                        &                                                                                          &                                                                             \\ \hline
\endfirsthead
%
\endhead
%
\hline
\endfoot
%
\endlastfoot
%
Einstieg                         & Begrüßung                                                                                                                   & Begrüßung                                                                            & Plenum                                                                                   &                                                                             \\
{[}Vorwissen bewusst machen{]}   & Frage: "Was ist eine Variable?"                                                                                             & Mögliche Antworten: "Ein Stück Speicher", "Eine Box"                                 &                                                                                          &                                                                             \\
                                 & Zeigt eine Variable in Scratch                                                                                              &                                                                                      & UG                                                                                       & SmartBoard                                                                  \\
15'                              & Frage: "Wo glaubt ihr benutzt Scratch intern Variablen?"                                                                    & Mögliche Antworten: "Bei Position", "Bei den Bildern"                                & Plenum                                                                                   &                                                                             \\ \hline
Hinführen                        & Zeigt Scratch Projekt:                                                                                                      &                                                                                      & UG                                                                                       & Smart Board                                                                 \\
                                 & Arbeitsauftrag: "Ihr sollt heute ein Adventure-Spiel mit Variablen machen.  Hier die Einführung."                           & Schüler*innen hören zu und planen innerlich ihr Projekt                              &                                                                                          &                                                                             \\
{[}Erarbeiten eines Prototyps{]} & Variable: Mood. Drei States "happy", "hungry" und "sad".                                                                    &                                                                                      &                                                                                          &                                                                             \\
                                 & Nach einer Zeit wird von "happy" auf "hungry" gestellt.                                                                     &                                                                                      &                                                                                          &                                                                             \\
10'                              & Brot verändert variable. Auch: Kommunikationsaspekt!                                                                        &                                                                                      &                                                                                          &                                                                             \\ \hline
Erarbeiten / Sichern I           & Zwei alternative Arbeitsaufträge.                                                                                           &                                                                                      & UG                                                                                       &                                                                             \\
                                 & "Baut den Prototypen nach und überlegt euch ein Verhalten für 'sad'"                                                        &                                                                                      &                                                                                          &                                                                             \\
5'                               & "Überlegt euch ein eigenes Adventure mit mindestens einer mood-variable, die von außen gesteuert werden kann"               &                                                                                      &                                                                                          &                                                                             \\
15'                              & Arbeitsphase an den Computern, zunächst in Einzelarbeit.                                                                    & Schüler*innen programmieren gezeigtes Beispiel nach oder entwickeln eigenes Szenario & EA                                                                                       & Computer / iPad                                                             \\ \hline
Erarbeiten / Sichern II          & Arbeitsauftrag: Stellt eure Implementierung euren Nachbarn vor                                                              &                                                                                      & PA                                                                                       & Computer / iPad                                                             \\
20'                              & Arbeitsauftrag: Entwickelt zusammen mit euren Sitznachbarn ein Spiel mit mindestens 2 Mood-variablen (z.B. extra Charakter) & Schüler*innen entwickeln spiele in Partnerarbeit                                     &                                                                                          &                                                                             \\ \hline
Ausstieg                         & Zwei Projekte kurz vorstellen                                                                                               & Zwei Gruppen stellen ihr Projekt kurz vor                                            & Plenum                                                                                   & SmartBoard / iPad                                                           \\
5'                               &                                                                                                                             &                                                                                      &                                                                                          &                                                                             \\ \hline
\end{tabularx}


\end{document}




